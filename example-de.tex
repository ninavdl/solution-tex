\documentclass[fleqn]{solution}

\usepackage{amssymb,amsmath,amsthm,mathtools}
\usepackage{color}
\usepackage{listings}
\usepackage{graphicx}
\usepackage{stmaryrd}
\usepackage{multicol}
\usepackage{enumitem}
\usepackage{lipsum}


\setgerman % Auskommentieren für Englisch
\setmodule{Beispielfach} % Modulname hier
\setsheetnum{1} % Blattnummer hier
\setgroupnum{5} % Gruppennummer hier (optional)

% Beliebig viele Autoren hier hinzufügen
\addauthor{Max Mustermann}{123456}
\addauthor{Anton Alliteration}{654321}


% Nummer der ersten Aufgabe (minus 1)
\setcounter{section}{0}

\begin{document}
	\maketitle
	
	\pointtable

	\begin{exercise}{Aufgabe mit bepunkteten Teilaufgaben}
		\begin{subexercise}[5]{Teilaufgabe mit Titel}
			Fubar bar. 

			\begin{subsubexercise}{Etwas}
				Unter-Teilaufgaben gehen natürlich auch.
			\end{subsubexercise} 

			\begin{subsubexercise}{}
				\lipsum[1]
			\end{subsubexercise}
		\end{subexercise}
		\begin{subexercise}[2]{}
			Dieser kleine Satz ist ganze zwei Punkte wert.
		\end{subexercise} 
	\end{exercise}
	 
	\begin{exercise}[2.5]{Aufgabe ohne Teilaufgaben}
		\lipsum[2-3]
	\end{exercise}
	
	\begin{exercise}[4]{Aufgabe mit unbepunkteten Teilaufgaben}
		Der Leser soll die tolle Kopfzeile zur Kenntnis nehmen.

		Diese Aufgabe gibt vier Punkte, aber es ist nicht genau gegeben,
		wie viele Punkte jede Teilaufgabe gibt.

		\begin{subexercise}{}
			Bla
		\end{subexercise}
		\begin{subexercise}{}
			Bla bla
		\end{subexercise}
		\begin{subexercise}{}
			Bla bla bla
		\end{subexercise}
	\end{exercise}
\end{document}
  